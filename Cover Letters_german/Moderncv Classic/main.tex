%% start of file `template.tex'.
%% Copyright 2006-2013 Xavier Danaux (xdanaux@gmail.com).
%
% This work may be distributed and/or modified under the
% conditions of the LaTeX Project Public License version 1.3c,
% available at http://www.latex-project.org/lppl/.


\documentclass[11pt,a4paper,sans]{moderncv}        % possible options include font size ('10pt', '11pt' and '12pt'), paper size ('a4paper', 'letterpaper', 'a5paper', 'legalpaper', 'executivepaper' and 'landscape') and font family ('sans' and 'roman')

% moderncv themes
\moderncvstyle{classic}                            % style options are 'casual' (default), 'classic', 'oldstyle' and 'banking'
\moderncvcolor{green}                              % color options 'blue' (default), 'orange', 'green', 'red', 'purple', 'grey' and 'black'
%\renewcommand{\familydefault}{\sfdefault}         % to set the default font; use '\sfdefault' for the default sans serif font, '\rmdefault' for the default roman one, or any tex font name
%\nopagenumbers{}                                  % uncomment to suppress automatic page numbering for CVs longer than one page

% character encoding
\usepackage[utf8]{inputenc}                       % if you are not using xelatex ou lualatex, replace by the encoding you are using
%\usepackage{CJKutf8}                              % if you need to use CJK to typeset your resume in Chinese, Japanese or Korean

% adjust the page margins
\usepackage[scale=0.75]{geometry}
%\setlength{\hintscolumnwidth}{3cm}                % if you want to change the width of the column with the dates
%\setlength{\makecvtitlenamewidth}{10cm}           % for the 'classic' style, if you want to force the width allocated to your name and avoid line breaks. be careful though, the length is normally calculated to avoid any overlap with your personal info; use this at your own typographical risks...

% personal data
\name{Ahmed}{Khalid}
\title{Resumé title}                               % optional, remove / comment the line if not wanted
\address{WichernStr 18}{91052 Erlangen}{Germany}% optional, remove / comment the line if not wanted; the "postcode city" and and "country" arguments can be omitted or provided empty
\phone[mobile]{+49 17669463160}                   % optional, remove / comment the line if not wanted
%\phone[fixed]{+2~(345)~678~901}                    % optional, remove / comment the line if not wanted
%\phone[fax]{+3~(456)~789~012}                      % optional, remove / comment the line if not wanted
\email{khalid.mahmoud92@hotmail.com}                               % optional, remove / comment the line if not wanted
%\homepage{www.johndoe.com}                         % optional, remove / comment the line if not wanted
%\extrainfo{additional information}                 % optional, remove / comment the line if not wanted
%\photo[64pt][0.4pt]{picture}                       % optional, remove / comment the line if not wanted; '64pt' is the height the picture must be resized to, 0.4pt is the thickness of the frame around it (put it to 0pt for no frame) and 'picture' is the name of the picture file
%\quote{Some quote}                                 % optional, remove / comment the line if not wanted

% to show numerical labels in the bibliography (default is to show no labels); only useful if you make citations in your resume
%\makeatletter
%\renewcommand*{\bibliographyitemlabel}{\@biblabel{\arabic{enumiv}}}
%\makeatother
%\renewcommand*{\bibliographyitemlabel}{[\arabic{enumiv}]}% CONSIDER REPLACING THE ABOVE BY THIS

% bibliography with mutiple entries
%\usepackage{multibib}
%\newcites{book,misc}{{Books},{Others}}
%----------------------------------------------------------------------------------
%            content
%----------------------------------------------------------------------------------
\begin{document}
%-----       letter       ---------------------------------------------------------
% recipient data
\recipient{xxxx GmbH}{Postfach 30 02 20, \\ xxxxx xxxxx, \\ Deutschland}
\date{Juli 23, 2017}
\opening{Sehr geeherter Herr Frank Hofmann,}
\closing{Mit freundlichen Grüßen,}
%\enclosure[Attached]{curriculum vit\ae{}}          % use an optional argument to use a string other than "Enclosure", or redefine \enclname
\makelettertitle

Mein Name ist Khalid und ich komme aus Ägypten. Ich habe meinen Master in Communication and Multimedia Engineering (CME) an der Friedrich-Alexander Universität in Erlangen im Juni 2017 absolviert.  Aktuell arbeite ich als Nachrichtentechnik Ingenieur am Fraunhofer IIS. Aktuell bin ich auf der Suche nach einem neuen Job und bewerbe mich hiermit auf die Stelle des "Forschungsingenieur/in für 5G Mobilfunk", welche ich auf ihrer Unternehmenswebsite gefunden habe.  

\textbf{xxxx} ist einer der größten internationalen Automobilzulieferer weltweit. Bei \textbf{xxxx} zu arbeiten wäre für mich persönlich ein großer Meilenstein in meiner Karrierelaufbahn. Wie Sie aus meinem Lebenslauf entnehmen können, konnte ich letzten Sommer in einem Praktikum in der Breitband Abteilung am Fraunhofer IIS praktische Erfahrungen sammeln. Während meines Praktikums konnte ich Wissen über den LTE Protokoll Stack gewinnen.  Ich hatte die Möglichkeit mit dem  Open Air Interface (OAI) LTE Simulator zu arbeiten, um kürzere Subframe-Länge (7 OFDM Symbole) anstatt von den jetzigen 14 OFDM Symbolen zu implementieren, um die Latenz zu verringern. Während meiner Masterarbeit habe ich einiges über System Modellierung gelernt, da ich in einer kurzen Zeit einen unabhängigen System Level Simulator mittels objektorientierter Programmierung in Matlab implementiert habe. Akademisch betrachtet decken mein Bachelor und mein Master Studium sowie meine bisherigen praktischen Erfahrungen, welche im Lebenslauf tiefer erklärt werden, die Anforderungen an die Stelle. 

Ich habe meinen Bachelor in Ägypten und mein Master Studium in Deutschland, was bedeutet, dass ich hinsichtlich verschiedener Arbeitsumgebungen sehr flexibel bin. 

Abschließend danke ich Ihnen für Ihre Zeit zur Durchsicht meiner Bewerbung und freue mich auf Ihre Antwort. 




\makeletterclosing

\end{document}


%% end of file `template.tex'.
