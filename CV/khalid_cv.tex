\documentclass[a4paper,10pt]{article}

%A Few Useful Packages
\usepackage{marvosym}
\usepackage{fontspec} 					%for loading fonts
\usepackage{xunicode,xltxtra,url,parskip} 	%other packages for formatting
\RequirePackage{color,graphicx}
\usepackage[usenames,dvipsnames]{xcolor}
\usepackage[big]{layaureo} 				%better formatting of the A4 page
% an alternative to Layaureo can be ** \usepackage{fullpage} **
%\usepackage{fullpage}
\usepackage{supertabular} 				%for Grades
\usepackage{titlesec}					%custom \section

%Setup hyperref package, and colours for links
\usepackage{hyperref}
\definecolor{linkcolour}{rgb}{0,0.2,0.6}
\hypersetup{colorlinks,breaklinks,urlcolor=linkcolour, linkcolor=linkcolour}

%FONTS
\defaultfontfeatures{Mapping=tex-text}
%\setmainfont[SmallCapsFont = Fontin SmallCaps]{Fontin}
%%% modified for Karol Kozioł for ShareLaTeX use
\setmainfont[
SmallCapsFont = Fontin-SmallCaps.otf,
BoldFont = Fontin-Bold.otf,
ItalicFont = Fontin-Italic.otf
]
{Fontin.otf}
%%%

%CV Sections inspired by: 
%http://stefano.italians.nl/archives/26
\titleformat{\section}{\Large\scshape\raggedright}{}{0em}{}[\titlerule]
\titlespacing{\section}{0pt}{3pt}{3pt}
%Tweak a bit the top margin
%\addtolength{\voffset}{-1.3cm}

%Italian hyphenation for the word: ''corporations''
\hyphenation{im-pre-se}

%-------------WATERMARK TEST [**not part of a CV**]---------------
\usepackage[absolute]{textpos}

\setlength{\TPHorizModule}{30mm}
\setlength{\TPVertModule}{\TPHorizModule}
\textblockorigin{2mm}{0.65\paperheight}
\setlength{\parindent}{0pt}

%--------------------BEGIN DOCUMENT----------------------
\begin{document}

%WATERMARK TEST [**not part of a CV**]---------------
%\font\wm=''Baskerville:color=787878'' at 8pt
%\font\wmweb=''Baskerville:color=FF1493'' at 8pt
%{\wm 
%	\begin{textblock}{1}(0,0)
%		\rotatebox{-90}{\parbox{500mm}{
%			Typeset by Alessandro Plasmati with \XeTeX\  \today\ for 
%			{\wmweb \href{http://www.aleplasmati.comuv.com}{aleplasmati.comuv.com}}
%		}
%	}
%	\end{textblock}
%}

\pagestyle{empty} % non-numbered pages

\font\fb=''[cmr10]'' %for use with \LaTeX command

%--------------------TITLE-------------
\par{\centering
		{\Huge Khalid \textsc{Mahmoud Mohamed Ahmed}
	}\bigskip\par}

%--------------------SECTIONS-----------------------------------
%Section: Personal Data
\section{Personal Data}
\begin{tabular}{rl}
    \textsc{Date of Birth:} & 23.07.1992   \\
		\textsc{Nationality:} & Egyptian  \\ 
    \textsc{Marital Status:} & Married  \\
    \textsc{Address:}   & Markweg 9,  \\
    & 91056 Erlangen, Germany  \\
    \textsc{Phone:}     & +4917669463160 \\
    \textsc{email:}     & engkhalid.mahmoud92@gmail.com \\
    &
\end{tabular}
\begin{textblock}{2.5}(4.5,-4.75)
	\includegraphics[width=4.5cm,height=4.2cm]{khalid.jpg}
\end{textblock}
%Section: Education
\section{Education}
\begin{tabular}{r|p{9cm}}	
\textsc{Oct.} 2014 - \textsc{June} 2017 & Masters of Science in Communication and Multimedia Engineering at the {\bf Friedrich-Alexander-University}, Erlangen, Germany (GPA 1.6/1.0). \\
& \\

 \textsc{July} 2014 & Bachelor of Science in Information Engineering and Technology with High Honours at the {\bf German University in Cairo}, Egypt (GPA 1.08/0.7)\\
& \\
 \textsc{Sept.} 2007 - \textsc{July} 2009 & {\bf Dr.Mahmoud Omar Secondary School}, Egypt
\end{tabular}

\section{Work Experience}
\begin{tabular}{r|p{9cm}}
	\textsc{June} 2017 - still running & Research engineer at {\bf Fraunhofer IIS}.\\
	& Implementing an LTE/5G uplink system level simulator using MATLAB object oriented programming. The simulator is used to test new multiple access schemes to reduce the latency for URLLC users in the uplink.\\
	& LTE/5G protocol stack development. Implementing short Transmission Time Interval (sTTI) feature in LTE release 15 to reduce latency. The development is done using Open Air Interface (OAI) platform. \\
	\textsc{January} 2015 - \textsc{April} 2016 & Student research assistant (Hiwi) in RFID Project at LIKE, {\bf Friedrich-Alexander-University}.\\
	& Implementing a maximum likelihood (ML) receiver for RFID tag reader using multiple receive antennas using MATLAB. \\
	& Validating the performance of ML receiver in a multiple input-multiple output (MIMO) double rayleigh backscatter channel. \\ 
\end{tabular}

%Section: Research
\section{Research}
\begin{tabular}{r|p{9cm}}
\textsc{Oct.} 2016 - \textsc{May.} 2017 & Master Thesis at the {\bf Friedrich-Alexander-University}, Erlangen, Germany in collaboration with {\bf Fraunhofer-Institut für Integrierte Schaltungen IIS}. "Uplink Multiple Access Schemes for Ultra-Low Latency Transmission", A system-level simulator is implemented using MATLAB object oriented programming simulation environment to test different proposals to guarantee fast access and high reliability to low latency users in LTE.\\
 \textsc{March} 2013 - \textsc{Sept.} 2013 & Bachelor Project at the {\bf Technical University in Ilmenau}, Germany. "Wireless Health Monitoring System Based on Fiber-Optic Sensors", a wireless portable system to measure the respiratory rate using a fiber Bragg grating (FBG) optical sensor is established. Analyzing and filtering the output data is explained and compared with the output data of a commercial piezoelectric sensor.\\
\end{tabular}
%Section: Work Experience at the top

\section{Internships}
\begin{tabular}{r|p{11cm}}
\textsc{May 2016 – Oct. 2016} & Internship at {\bf Fraunhofer IIS}. The task was to develop and enhance the {OAI} simulation environment to allow for shorter {TTI} in the current LTE protocol stack on the physical layer in the downlink using {\bf C} programming language. The task was a step towards development of 5G cellular stack. A 7-OFDM symbol downlink {TTI} was developed and tested using OAI simulation environment. \\
& \\ 
\textsc{Oct. 2012 – Jan. 2013} & Junior teaching assistant at the {\bf German university in Cairo} teaching CSIS104 for pharmacy students and CSEN102 for engineering students. \\
& \\
\textsc{August} 2012 & Wireless internship at the {\bf German University in Cairo} learning to work on tinyos to program mib510 motes using \textbf{nesC} programing language, then making a simple application about indoor localization using Finger Printing algorithm.\\
& \\
 \textsc{July} 2012 & Radio Frequency (RF) intern at the {\bf German University in Cairo}, designing and simulating couplers, filters and phase shifter using  Computer Simulation Technology (CST), then working on paper "wide band 180 hybrid coupler" and modifying its design to give better results.\\ 
& \\
\textsc{July} 2011 & Summer Internship at the {\bf Biomedical Institute, Technical University in Ilmenau}, Germany, designing a flash control system by programming Texas micro controller using \textbf{C} language.\\ 
& \\
 \textsc{July} 2010  & Summer Internship at the {\bf Biomedical Institute, Technical University in Ilmenau}, Germany, designing and fabricating simple printed circuit boards (PCBs).\\

\end{tabular}





%Section: Languages
\section{Languages}
\begin{tabular}{rl}
\textsc{Arabic:}&Mother Tongue\\
\textsc{English:}&Fluent\\
\textsc{German:}&B1.2\\
\end{tabular}

\section{Computer Skills}
\begin{tabular}{rl}
Very Good Knowledge:& \textsc{MATLAB} \\
Intermediate Knowledge:& \textsc{CST - Computer Simulation Technology}, \textsc{java} and {\fb \LaTeX}\setmainfont[SmallCapsFont=Fontin-SmallCaps.otf]{Fontin.otf}\\
Basic Knowledge:& C, GIT and Mathematica
\end{tabular}

\section{Interests and Activities}
\subsection*{Research Interests}
Digital Modulation, MIMO, Digital Signal Proccessing, LTE/NR and Information Theory.\\
%\subsection*{Activities}
%Member in AYB (A'lashanek Ya Balady) NGO charity and Bdaya NGO charity. \\
%Football, Tennis and Cycling


%\newpage
%\hypertarget{gmat}{\textsc{Gmat}\setmainfont{LMRoman10 Regular}\textregistered\setmainfont[SmallCapsFont=Fontin-SmallCaps]{Fontin-Regular}}

%\XeTeXpdffile ''GMAT.pdf'' page 1 scaled 800

\end{document}
